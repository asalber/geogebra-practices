% !TEX root = ../geogebra-practices.tex
% Author: Alfredo Sánchez Alberca (asalber@ceu.es)
\chapter{Elementary functions}

% \section{Fundamentos teóricos}

% En esta práctica se introducen los conceptos básicos sobre funciones reales de variable real, esto es, funciones
% \[f:\mathbb{R}\rightarrow \mathbb{R}.\]

% \subsection{Dominio e imagen}

% El \emph{Dominio} de la función $f$ es el conjunto de los números reales $x$ para los que existe $f(x)$ y se designa mediante $\dom f$.

% La \emph{Imagen} de $f$ es el conjunto de los números reales $y$ para los que existe algún $x\in \mathbb{R}$ tal que $f(x)=y$, y se denota por $\im f$.


% \subsection{Signo y crecimiento}
% El \emph{signo} de la función es positivo $(+)$ en los valores de $x$ para los que $f(x)>0$ y negativo $(-)$ en los que $f(x)<0$.
% Los valores de $x$ en los que la función se anula se conocen como \emph{raíces} de la función.

% Una función $f(x)$ es \emph{creciente} en un intervalo $I$ si $\forall\, x_1, x_2 \in I$ tales que $x_1<x_2$ se verifica que $f(x_1)\leq f(x_2)$.

% Del mismo modo, se dice que una función $f(x)$ es \emph{decreciente} en un intervalo $I$ si $\forall\, x_1, x_2 \in I$ tales que $x_1<x_2$ se verifica que $f(x_1)\geq f(x_2)$. En la figura~\ref{g:crecimiento} se muestran estos conceptos.

% \begin{figure}[h!]
% 	\centering \subfigure[Función creciente.] {\label{g:funcion_creciente}
% 		\scalebox{1}{\input{img/funciones_elementales/funcion_creciente}}}\qquad
% 	\subfigure[Función decreciente.]{\label{g:funcion_decreciente}
% 		\scalebox{1}{\input{img/funciones_elementales/funcion_decreciente}}}
% 	\caption{Crecimiento de una función.}
% 	\label{g:crecimiento}
% \end{figure}


% \subsection{Extremos Relativos}
% Una función $f(x)$ tiene un \emph{máximo relativo} en $x_0$ si existe un entorno $A$ de $x_0$ tal que $\forall x \in A$
% se verifica que $f(x)\leq f(x_0)$.

% Una función $f(x)$ tiene un \emph{mínimo relativo} en $x_0$ si existe un entorno $A$ de $x_0$ tal que $\forall x\in A$
% se verifica que $f(x)\geq f(x_0)$.

% Diremos que la función $f(x)$ tiene un \emph{extremo relativo} en un punto si tiene un \emph{máximo o mínimo relativo}
% en dicho punto. Estos conceptos se muestran en la figura~\ref{g:extremos}.

% \begin{figure}[h!]
% 	\centering \subfigure[Máximo relativo.] {\label{g:maximo}
% 		\scalebox{1}{\input{img/funciones_elementales/maximo}}}\qquad
% 	\subfigure[Mínimo relativo.]{\label{g:minimo}
% 		\scalebox{1}{\input{img/funciones_elementales/minimo}}}
% 	\caption{Extremos relativos de una función.}
% 	\label{g:extremos}
% \end{figure}

% Una función $f(x)$ está \emph{acotada superiormente} si $\exists K\in\mathbb{R}$ tal que $f(x)\leq K$ $\forall x \in \dom f$. Análogamente, se dice que una función $f(x)$ está \emph{acotada inferiormente} si $\exists K\in\mathbb{R}$ tal que $f(x)\geq K$ $\forall x \in \dom f$.

% Una función $f(x)$ está \emph{acotada} si lo está superior e inferiormente, es decir si $\exists K\in\mathbb{R}$ tal que $|f(x)|\leq K$ $\forall x \in \dom f$.


% \subsection{Concavidad}

% De forma intuitiva se puede decir que una función $f(x)$ es \emph{cóncava} en un intervalo $I$ si $\forall\, x_1, x_2
% \in I$, el segmento de extremos $(x_1,f(x_1))$ y $(x_2,f(x_2))$ queda por encima de la gráfica de $f$.

% Análogamente se dirá que es \emph{convexa} si el segmento anterior queda por debajo de la gráfica de $f$.

% Diremos que la función $f(x)$ tiene un \emph{punto de inflexión} en $x_0$ si en ese punto la función pasa de cóncava a
% convexa o de convexa a cóncava. Estos conceptos se ilustran en la figura~\ref{g:concavidad}.

% \begin{figure}[h!]
% 	\centering \subfigure[Función cóncava.] {\label{g:funcion_convexa}
% 		\scalebox{1}{\input{img/funciones_elementales/funcion_convexa}}}\qquad
% 	\subfigure[Función convexa.]{\label{g:funcion_concava}
% 		\scalebox{1}{\input{img/funciones_elementales/funcion_concava}}}
% 	\caption{Concavidad de una función.}
% 	\label{g:concavidad}
% \end{figure}

% \subsection{Asíntotas}

% La recta $x=a$ es una \emph{asíntota vertical} de la función $f(x)$ si al menos uno de los límites laterales de $f(x)$ cuando $x$ tiende hacia $a$ es $+\infty$ o $-\infty$, es decir cuando se verifique alguna de las siguientes igualdades
% \[
% 	\ \lim_{x\rightarrow a^{+}}f(x)=\pm\infty   \quad \textrm{o} \quad
% 	\lim_{x\rightarrow a^{-}}f(x)=\pm\infty
% \]

% La recta $y=b$ es una \emph{asíntota horizontal} de la función $f(x)$ si alguno de los límites de $f(x)$ cuando $x$ tiende hacia $+\infty$ o $-\infty$ es igual a $b$, es decir cuando se verifique
% \[
% 	\ \lim_{x\rightarrow -\infty }f(x)=b    \quad \textrm{o} \quad
% 	\ \lim_{x\rightarrow +\infty }f(x)=b
% \]

% La recta $y=mx+n$ es una \emph{asíntota oblicua} de la función $f(x)$ si alguno de los límites de $f(x)-(mx+n)$ cuando $x$ tiende hacia $+\infty$ o $-\infty$ es igual a 0, es decir si

% \[
% 	\ \lim_{x\rightarrow -\infty }{(f(x)-mx)}=n    \quad \textrm{o} \quad
% 	\ \lim_{x\rightarrow +\infty }{(f(x)-mx)}=n
% \]

% En la figura~\ref{g:asintotas} se muestran los distintos tipos de asíntotas.

% \begin{figure}[h!]
% 	\centering \subfigure[Asíntota horizontal y vertical.] {\label{g:asintotahorizontalyvertical}
% 		\scalebox{1}{\input{img/funciones_elementales/asintota_vertical}}}\qquad\qquad
% 	\subfigure[Asíntota vertical y oblicua.]{\label{g:asintotaoblicua}
% 		\scalebox{1}{\input{img/funciones_elementales/asintota_oblicua}}}
% 	\caption{Tipos de asíntotas de una función.}
% 	\label{g:asintotas}
% \end{figure}


% \subsection{Periodicidad}
% Una función $f(x)$ es \emph{periódica} si existe $h\in\mathbb{R^{+}}$ tal que \[f(x+h)=f(x)\  \forall x\in \dom f\] siendo el período $T$ de la función, el menor valor $h$ que verifique la igualdad anterior.

% En una función periódica, por ejemplo $f(x)=A\sin(wt)$, se denomina \emph{amplitud} al valor de $A$, y es la mitad de la diferencia entre los valores máximos y mínimos de la función. En la figura~\ref{g:periodoyamplitud} se ilustran estos conceptos.

% \begin{figure}[h!]
% 	\centering
% 	\scalebox{0.8}{\input{img/funciones_elementales/funcion_periodica}}
% 	\caption{Periodo y amplitud de una función periódica.}
% 	\label{g:periodoyamplitud}
% \end{figure}

% \clearpage
% \newpage

\section{Solved exercises}

\begin{enumerate}[leftmargin=*]
\item Plot the graph of the function
      \[
      f(t)=\frac{t^{4} +19t^{2} - 5}{t^{4} +9t^{2} - 10}.
      \]
      and determine, looking at the graph, the following:

      \begin{enumerate}
      \item Domain.
            \begin{indication}
            \begin{enumerate}
            \item Enter the expression \command{(t\^{}4+19t\^{}2-5)/(t\^{}4+9t\^{}2-10)} in the \field{Input Bar} of the \field{CAS View} and activate the \field{Graphics View}.
            \item To determine the domain look at the values of $x$ where the function does exists, that is, where there is graph.
            \item Remember that, for this and the other parts of this exercise, we pretend to get approximate conclusions looking at the graph of the function.
            \end{enumerate}
            \end{indication}

      \item Image.
            \begin{indication}
            Look at the values of $y$ that are output of the function, that is, where there is graph.
            \end{indication}

      \item Asymtotes.
            \begin{indication}
            Look at the lines (horizontal, vertical or oblique) where the graph approaches (the distance between the graph and the line tends to zero as they tend to infinity).
            \end{indication}

      \item Roots.
            \begin{indication}
            Look at the values of $x$ where the graph cuts the horizontal axis.
            \end{indication}

      \item Sign.
            \begin{indication}
            Look at the values of $x$ where the graph is over the horizontal axis (positive) and where it is under the horizontal axis (negative).
            \end{indication}

      \item Growth.
            \begin{indication}
            Look at the values of $x$ where $y$ increases when $x$ increases (increasing) and the values where $y$ decreases when $x$ increases (decreasing).
            \end{indication}

      \item Concavity.
            \begin{indication}
            Look at the values of $x$ where the curvature of the graph is up $\cup$ (concave up or convex) and where the curvature is $\cap$ (concave down or simply concave).
            \end{indication}

      \item Local extrema.
            \begin{indication}
            Look at the values of $x$ where the graph has a peak (relative maximum) and where the graph has a valley (relative minimum).
            \end{indication}

      \item Inflection points.
            \begin{indication}
            Look at the values of $x$ where the curvature changes continuously.
            \end{indication}
      \end{enumerate}

\item Plot the graphs of the exponential functions $2^x$, $e^x$, $0.7^x$, $0.5^x$.
      Looking at the graphs of the functions, determine for which values of the base the exponential function increases and for which ones decreases.

      \begin{indication}
      \begin{enumerate}
      \item Enter the expression \command{2\^{}x} in the \field{Input Bar} of the \field{CAS View}.
      \item Enter the expression \command{e\^{}x} in the \field{Input Bar}.
      \item Enter the expression \command{0.7\^{}x} in the \field{Input Bar}.
      \item Enter the expression \command{0.5\^{}x} in the \field{Input Bar}.
      \item The exponential function $a^x$ increases for $a>1$ and decreases for $0<a<1$.
      \end{enumerate}
      \end{indication}

\item Plot the graphs of the following functions and determine their periods and amplitudes.
      \begin{enumerate}
      \item $\sin{x}$, $\sin{x}+2$, $\sin{(x+2)}$.
            \begin{indication}
            \begin{enumerate}
            \item Enter the expression \command{sen(x)} en la in the \field{Input Bar} of the \field{CAS View}.
            \item Enter the expression \command{sen(x)+2} in the \field{Input Bar}.
            \item Enter the expression \command{sen(x+2)} in the \field{Input Bar}.
            \item When a value is added to the sine or to the argument of the sine, the period and the amplitude does not change.
            \end{enumerate}
            \end{indication}
      \item $\sin{2x}$, $2\sin{x}$, $\sin\frac{x}{2}$.
            \begin{indication}
            \begin{enumerate}
            \item Enter the expression \command{sen(2x)} in the \field{Input Bar}.
            \item Enter the expression \command{2sen(x)} in the \field{Input Bar}.
            \item Enter the expression \command{sen(x/2)} in the \field{Input Bar}.
            \item When the sine is mulitiplied by a value, the amplitude is multiplied by that value, and when the argument of the sine is multiplied by a value, the period is divided by that value.
            \end{enumerate}
            \end{indication}
      \end{enumerate}


\item Plot the graph of the piecewise function
      \[
      \ f(x)=\left\{
      \begin{array}{cl}
      -2x   & \hbox{si $x\leq0$;} \\
      x^{2} & \hbox{si $x>0$.}    \\
      \end{array}
      \right.
      \]

      \begin{indication}
      Enter the expression \command{If(x<=0, -2x, x\^{}2)} in the \field{Input Bar}.
      \end{indication}
\end{enumerate}


\section{Proposed exercises}
\begin{enumerate}[leftmargin=*]
\item Plot the graphs of the following functions and determine their domains looking at their graphs.

      \begin{enumerate}
      \item $f(x)=\dfrac{x^2+x+1}{x^3-x}$
      \item $g(x)=\sqrt{x^4-1}$.
      \item $h(x)=\cos\left(\frac{x+3}{x^2+1}\right)$.
      \item $l(x)=\arcsin\left(\frac{x}{1+x}\right)$.
      \end{enumerate}

\item Consider the function
      \[
      f(x)=\frac{x^3+x+2}{5x^3-9x^2-4x+4}.
      \]

      Plot the function and determine looking at its graph:
      \begin{enumerate}
      \item Domain
      \item Image
      \item Asymptotes
      \item Zeros
      \item Sign
      \item Continuity
      \item Increasing and decreasing
      \item Concavity
      \item Relative extrema
      \item Inflection points
      \end{enumerate}

\item Plot in the same graphic window the graphs of the functions $\log_{10}x$, $\log_{2}x$, $\log x$, $\log_{0.5}x$.
      \begin{enumerate}
      \item Looking at their graphs determine which funtions are increasing and which ones are decreasing.
      \item Deduce for what values of $a$ the function $\log_ax$ is increasing and for what values it is decreasing?
      \end{enumerate}

\item Plot the following functions and complete the following sentences with the word equal or the number of times that is lower or greater in any case.
      \begin{enumerate}
      \item The function $\cos(2x)$ has a period ............ than the function $\cos{x}$.
      \item The function $\cos(2x)$ has an amplitude ............ than the function $\cos{x}$.
      \item The function $\cos(x/2)$ has period ............ than the function $\cos(3x)$.
      \item The function $\cos(x/2)$ has an amplitude ............ than the function $\cos(3x)$.
      \item The function $3\cos(2x)$ has a period ............ than the function $\cos(x/2)$.
      \item The function $3\cos(2x)$ has an amplitude ............ than the function $\cos(x/2)$.
      \end{enumerate}

\item Find the solutions of the equation $e^{-1/x}=\dfrac{1}{x}$  graphically.

\item Plot the graph of the function
      \[
      f(x)=
      \begin{cases}
      x^3   & \mbox{if $x<0$}    \\
      e^x-1 & \mbox{if $x\geq0$}
      \end{cases}
      \]
\end{enumerate}
