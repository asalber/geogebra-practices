% !TEX root = ../geogebra-practices.tex
% Author: Alfredo Sánchez Alberca (asalber@ceu.es)
\chapter{Derivatives of functions of one variable}

% \section{Fundamentos teóricos}
% El concepto de derivada es uno de los más importantes del Cálculo
% pues resulta de gran utilidad en el estudio de funciones y tiene
% multitud de aplicaciones. En esta práctica introducimos este
% concepto y presentamos algunas de sus aplicaciones, tanto en
% funciones de una como de varias variables.

% \subsection{Tasas de variación media e instantánea. La derivada}
% Cuando queremos conocer la variación que experimenta una función
% real $f(x)$ en un intervalo $[a,b]$, se calcula la diferencia
% $f(b)-f(a)$ que se conoce como \emph{incremento} de $f$, y se nota
% $\Delta f[a,b]$, aunque, a veces, simplemente se escribe $\Delta f$.

% En muchas otras ocasiones veces resulta importante comparar la
% variación que experimenta la función $f$ con relación a la variación
% que experimenta su argumento $x$ en un intervalo $[a,b]$. Si tenemos
% en cuenta que $b=a+\Delta x$, esto viene dado por la \emph{tasa de
% variación media}, que se define como:

% \[
% \textrm{TVM} f[a,b]=\textrm{TVM} f[a,a+\Delta x]=\frac{\Delta
% f}{\Delta x}=\frac{f(b)-f(a)}{b-a}=\frac{f(a+\Delta x )-f(a)}{\Delta
% x}. 
% \]

% También resulta muy común llamar a $\Delta x$ con la letra $h$, por
% lo que la expresión anterior queda de la forma:

% \[
% \textrm{TVM} f[a,b]=\textrm{TVM} f[a,a+h]=\frac{\Delta f}{\Delta
% x}=\frac{f(a+h)-f(a)}{h}.
% \]

% Desde el punto de vista geométrico, la tasa de variación media de
% $f$ en el intervalo $[a , a+\Delta x]$ es la pendiente de la recta
% secante a $f$ en los puntos $(a , f(a))$ y $(a+\Delta x, f(a+\Delta
% x))$, tal y como se muestra en la figura~\ref{g:secante}.

% \begin{figure}[h!]
% \begin{center}
% \scalebox{1}{\input{img/derivadas/secante}}
% \caption{La tasa de variación media como la pendiente de la recta
% secante a una función en dos puntos.} \label{g:secante}
% \end{center}
% \end{figure}


% Y, a veces, incluso más importante que la tasa de variación media,
% es estudiar la tasa de variación que experimenta la función, no en
% un intervalo, sino en un punto, tomando para ello límites cuando el
% incremento en la variable independiente tiende 0. Definimos la tasa
% de variación de una función en un punto $a$, o también tasa de
% variación instantánea, a partir de la tasa de variación media de la
% función en el intervalo $[a,a+\Delta x]$. Dicha tasa, si existe,
% recibe el nombre de \emph{derivada} de la función real $f(x)$ en un
% punto $a\in \mathbb{R}$, y se nota como $f'(a)$, o bien
% $\dfrac{df}{dx}(a)$:

% \[
% f'(a)=\dfrac{df}{dx}(a)= \lim_{\Delta x\rightarrow 0}\frac{\Delta
% f}{\Delta x}=\lim_{\Delta x\rightarrow 0}\frac{f(a+\Delta
% x)-f(a)}{\Delta x}=\lim_{h\rightarrow 0}\frac{f(a+h)-f(a)}{h}.
% \]

% Cuando este límite existe, se dice que la función $f$ es
% \emph{derivable} o \emph{diferenciable} en el punto $a$.

% Geométricamente, $f'(a)$ es la pendiente de la recta tangente a la
% curva de $f(x)$ en el punto $(a,f(a))$, tal y como se aprecia en la
% figura~\ref{g:tangente}.

% \begin{figure}[h!]
% \begin{center}
% \scalebox{1}{\input{img/derivadas/tangente}}
% \caption{La derivada como la pendiente de la recta tangente a una
% función en un punto.} \label{g:tangente}
% \end{center}
% \end{figure}

% \subsubsection*{Recta tangente y normal a una función en un punto}
% De la gráfica anterior, fácilmente se deduce que la ecuación de la
% recta tangente a una función $f(x)$ en el punto $(a,f(a))$ es:
% \[
% y=f(a)+f'(a)(x-a).
% \]

% Y teniendo en cuenta que la pendiente de la recta normal (recta
% perpendicular a la recta tangente) es la inversa cambiada de signo,
% la ecuación de la recta normal a $f(x)$ en el punto $(a,f(a))$ es:
% \[
% y=f(a)-\frac{1}{f'(a)}(x-a).
% \]

% \subsection{Función derivada y derivadas sucesivas}

% El límite que nos sirve para calcular the derivative of una función en
% un punto, define una nueva función $f'$ cuyo dominio está formado
% por los puntos en los que $f$ es diferenciable. La función $f'(x)$,
% o también $\dfrac{df}{dx}$, se llama \emph{primera derivada} de $f$.

% Puesto que $f'$ es una función, puede derivarse a su vez, y la
% primera derivada de $f'$ se conoce como segunda derivada de $f$, y
% se nota $f''(x)$ o $\dfrac{d^2f}{dx^2}$. Análogamente, la
% \emph{$n$-ésima derivada} de $f$, designada por $f^{(n}$ o
% $\dfrac{d^nf}{dx^n}$, es la primera derivada de $f^{(n-1}$, para
% $n=2,3,\ldots$, es decir
% \[
% \frac{d^nf}{dx^n}=\frac{d}{dx}\left(\frac{d^{n-1}f}{dx^{n-1}}\right)\
% n=2,3,\ldots
% \]


% \subsection{Estudio del crecimiento de una función}
% Una función $f(x)$ es \emph{creciente} en un intervalo $I$ si
% $\forall\, x_1, x_2 \in I$ tales que $x_1<x_2$ se verifica que
% $f(x_1)\leq f(x_2)$.

% Del mismo modo, se dice que una función $f(x)$ es \emph{decreciente}
% en un intervalo $I$ si $\forall\, x_1, x_2 \in I$ tales que
% $x_1<x_2$ se verifica que $f(x_1)\geq f(x_2)$. En la
% figura~\ref{g:crecimiento_derivada} se muestran estos conceptos.

% \begin{figure}[h!]
% \centering \subfigure[Función creciente.] {\label{g:funcion_creciente}
% \scalebox{1}{\input{img/funciones_elementales/funcion_creciente}}}\qquad
% \subfigure[Función decreciente.]{\label{g:funcion_decreciente}
% \scalebox{1}{\input{img/funciones_elementales/funcion_decreciente}}}
% \caption{Crecimiento de una función.}
% \label{g:crecimiento_derivada}
% \end{figure}

% Si $f$ es una función derivable en el intervalo $I$, el signo de la
% derivada puede utilizarse para estudiar el crecimiento de la función
% ya que se cumple:
% \begin{itemize}
% \item $f$ es creciente en $x_0\in I$, si y sólo si, $f'(x_0)\geq 0$.
% \item $f$ es decreciente en $x_0\in I$, si y sólo si, $f'(x_0)\leq 0$.
% \end{itemize}

% Desde el punto de vista geométrico, esto es evidente, ya en los
% intervalos donde $f$ es creciente, cualquier recta tangente tiene
% pendiente positiva, mientras que en los intervalos donde $f$ es
% decreciente, las tangentes tienen pendiente negativa, tal y como se
% observa en la figura~\ref{g:crecimiento_derivada}.

% \subsection{Determinación de los extremos relativos}
% Una función $f(x)$ tiene un \emph{máximo relativo} en $x_0$ si
% existe un entorno $A$ de $x_0$ tal que $\forall x \in A$ se verifica
% que $f(x)\leq f(x_0)$.

% Una función $f(x)$ tiene un \emph{mínimo relativo} en $x_0$ si
% existe un entorno $A$ de $x_0$ tal que $\forall x\in A$ se verifica
% que $f(x)\geq f(x_0)$.

% Diremos que la función $f(x)$ tiene un \emph{extremo relativo} en un
% punto si tiene un \emph{máximo o mínimo relativo} en dicho punto.

% Cuando $f$ es una función continua, entonces también se puede
% definir un extremo relativo como aquel punto donde cambia el
% crecimiento de la función. Así, un máximo relativo es un punto donde
% la función pasa de ser creciente a ser decreciente, y un mínimo
% relativo es un punto donde la función pasa de ser decreciente a ser
% creciente, tal y como se muestra en la figura~\ref{g:extremos_derivada}.

% \begin{figure}[h!]
% \centering \subfigure[Máximo relativo.] {\label{g:maximo}
% \scalebox{1}{\input{img/funciones_elementales/maximo}}}\qquad
% \subfigure[Mínimo relativo.]{\label{g:minimo}
% \scalebox{1}{\input{img/funciones_elementales/minimo}}}
% \caption{Extremos relativos de una función.}
% \label{g:extremos_derivada}
% \end{figure}

% Si $f$ tiene un extremo relativo en un punto $x_0$ y existe la
% derivada en dicho punto, entonces se cumple que $f'(x_0)=0$, es
% decir, la tangente a la gráfica de $f$ en dicho punto es horizontal
% (figura~\ref{g:extremos_derivada}). El recíproco no es cierto en general, de
% modo que esta es una condición necesaria pero no suficiente. No
% obstante, si $f$ es una función derivable en un intervalo $I$,
% podemos utilizar esta propiedad para detectar los puntos entre los
% que se encontrarán los extremos relativos del intervalo $I$. Los
% puntos donde se anula la primera derivada, se conocen como
% \emph{puntos críticos} y serán candidatos a extremos. Una vez
% detectados los puntos críticos, para ver si se trata de un extremo
% relativo o no, basta con estudiar el crecimiento de la función a la
% izquierda y a la derecha del punto tal y como se indicaba en la
% sección anterior. Resumiendo, si $f'(x_0)=0$, entonces:
% \begin{itemize}
% \item Si existe un $\delta>0$ tal que $f'(x)>0\ \forall\, x\in (x_0-\delta,x_0)$ (derivada positiva a la izquierda de $x_0$) y $f'(x)<0\ \forall\, x\in (x_0,x_0+\delta)$ (derivada negativa a la derecha de $x_0$), $x_0$ es un máximo relativo.

% \item Si existe un $\delta>0$ tal que $f'(x)<0\ \forall\, x\in (x_0-\delta,x_0)$ (derivada negativa a la izquierda de $x_0$) y $f'(x)>0\ \forall\, x\in (x_0,x_0+\delta)$ (derivada positiva a la derecha de $x_0$), $x_0$ es un mínimo relativo.

% \item En cualquier otro, $x_0$ es un \emph{punto de inflexión}.
% \end{itemize}

% \subsection{Estudio de la concavidad de una función}
% Se dice que una función $f(x)$ es \emph{cóncava} en un intervalo $I$
% si $\forall\, x_1, x_2 \in I$, el segmento de extremos
% $(x_1,f(x_1))$ y $(x_2,f(x_2))$ queda por encima de la gráfica de
% $f$.

% Análogamente se dirá que es \emph{convexa} si el segmento anterior
% queda por debajo de la gráfica de $f$.

% Diremos que la función $f(x)$ tiene un \emph{punto de inflexión} en
% $x_0$ si en ese punto la función pasa de cóncava a convexa o de
% convexa a cóncava. Estos conceptos se ilustran en la
% figura~\ref{g:concavidad_derivada}.

% \begin{figure}[h!]
% \centering \subfigure[Función cóncava.] {\label{g:funcion_concava_arriba}
% \scalebox{1}{\input{img/funciones_elementales/funcion_concava_arriba}}}\qquad
% \subfigure[Función convexa.]{\label{g:funcion_concava_abajo}
% \scalebox{1}{\input{img/funciones_elementales/funcion_concava_abajo}}}
% \caption{Concavidad de una función.}
% \label{g:concavidad_derivada}
% \end{figure}


% Si $f$ es una función derivable en el intervalo $I$, el signo de la
% segunda derivada puede utilizarse para estudiar la concavidad de la
% función ya que se cumple:
% \begin{itemize}
% \item $f$ es cóncava en $x_0\in I$, si y sólo si, $f''(x_0)\geq 0$.
% \item $f$ es convexa en $x_0\in I$, si y sólo si, $f''(x_0)\leq 0$.
% \end{itemize}

% \newpage

\section{Solved exercises}
\begin{enumerate}[leftmargin=*]
% \item Given the function $f(x)=\dfrac{x^3+x^2-2x-2}{x+3}$, se pide:
% \begin{enumerate}
% \item Calcular la tasa de variación media de $f$ en los intervalos $[-1,3]$, $[-1,0]$ y $[-1,-0.5]$, y calcular las correspondientes rectas
% secantes.
% \begin{indication}
% \begin{enumerate}
% \item To compute la tasa de variación media, definir previamente la función, y luego, por ejemplo para el intervalo $[-1,3]$, aplicamos
% la fórmula vista en la teoría:
% \[
% \textrm{TVM} f[-1,3]=\frac{\Delta y}{\Delta
% x}=\frac{f(3)-f(-1)}{3-(-1)}.
% \]
% \item To compute la ecuación de la recta secante, podemos utilizar, por ejemplo, la ecuación de la recta de la que conocemos un punto por
% el que pasa, $(x_0,y_0)$ y su pendiente, $m$:
% \[
% y - y_0  = m\left( {x - x_0 } \right)
% \]
% En nuestro caso, para el primero de los intervalos considerados, el punto puede ser, por ejemplo el $(-1,f(-1))$; y la pendiente viene dada
% por la tasa de variación media de la función en dicho intervalo. Es decir, la recta que buscamos tendrá como ecuación:
% \[
% y - f( - 1) = \textrm {TVM} f[ - 1,3]\left( {x - ( - 1)} \right)
% \]
% \item Después de calcular la ecuación de la recta secante, podemos comprobar que la misma corta a la función en los puntos adecuados sin más
% que representar en la misma gráfica tanto $f$ como la recta calculada.
% \end{enumerate}
% \end{indication}
% 
% 
% \item Calcular la tasa de variación instantánea de $f$ en el punto $-1$ haciendo uso de límites, y calcular la correspondiente recta
% tangente.
% \begin{indication}
% \begin{enumerate}
% \item Como ya sabemos por la teoría, las tasa de variación instantánea de la función en un punto dado si existe, recibe el nombre de
% derivada de la función en el punto, y se calcula mediante the limit:
% \[
% f'(a) = \mathop {\lim }\limits_{h \to 0} \frac{{f(a + h) - f(a)}}
% {h}
% \]
% en donde, por aligerar la notación, hemos llamado $h$ a lo que en la teoría denominábamos $\Delta x$.
% 
% Por lo tanto, para calcular the derivative of la función $f$ en $a=-1$ mediante la definición, procedemos con:
% \[
% f'(-1) = \mathop {\lim }\limits_{h \to 0} \frac{{f(-1 + h) - f(-1)}}
% {h}
% \]
% To compute the limit, podemos utilizar el botón \button{Calcular un límite} de la barra de botones.
% \item Para el cálculo de la recta tangente, de nuevo sabemos que la misma pasa por el punto $(-1, f(-1))$, y que su pendiente vale $f'(-1)$.
% Por lo tanto su ecuación es:
% \[
% y - f( - 1) = f'(-1)\left( {x - ( - 1)} \right)
% \]
% \item De nuevo, conviene representar en la misma gráfica tanto la función como la recta tangente en el punto considerado, para comprobar que
% los cálculos han sido los correctos.
% \end{enumerate}
% \end{indication}
% \end{enumerate}


\item Plot the graph of the following functions and study its differentiability at the points given using limits.
      \begin{enumerate}
      \item $f(x)=|x-1|$ en $x=1$.

            \begin{indication}
            \begin{enumerate}
            \item To plot the graph, enter the function \command{f(x):=|x-1|} in the \field{Input Bar} of the \field{CAS View} and activate the \field{Graphics View}.
            \item To compute the derivative of $f$ to the left at $x=1$ enter the command \command{LimitBelow((f(1+h)-f(1))/h, 0)} in the \field{Input Bar}.
            \item To compute the derivative of $f$ to the right at $x=1$ enter the command \command{LimitAbove((f(1+h)-f(1))/h, 0)} in the \field{Input Bar}.
            \item Since $lim_{x\rightarrow 0^-}\frac{f(1+h)-f(1)}{h}=-1$ is not equal to $lim_{x\rightarrow 0^+}\frac{f(1+h)-f(1)}{h}=1$, there is no derivative at point $x=1$.
            \end{enumerate}
            \end{indication}

      \item $g(x)=\left\{
            \begin{array}{ll}
            x \sin\dfrac{1}{x}, & \hbox{if $x\neq 0$;} \\
            0,                  & \hbox{if $x=0$.}     \\
            \end{array}
            \right. \quad \textrm{en $x=0$.}$

            \begin{indication}
            \begin{enumerate}
            \item To plot the graph, enter the function \command{g(x):=If(x!=0, sen(1/x), 0)} in the \field{Input Bar} of the \field{CAS View}.
            \item To compute the derivative of $g$ to the left at $x=0$ enter the command \command{LimitBelow((g(h)-g(0))/h, 0)} in the \field{Input Bar}.
            \item To compute the derivative of $g$ to the right $x=0$ enter the command \command{LimitAbove((g(h)-g(0))/h, 0)} in the \field{Input Bar}.
            \item Since none of these limits exists, $g$ is not differentiable at $x=0$.
            \end{enumerate}
            \end{indication}
      \end{enumerate}


\item Compute the derivatives of the following function until order 4 and try to infer the formula for the derivative of order $n$ in each case.
      \begin{enumerate}
      \item  $f(x)=a^x\log(a)$.
            \begin{indication}
            \begin{enumerate}
            \item Enter the function \command{f(x):=a\^{}xlog(a)} in the \field{Input Bar} of the \field{CAS View}.
            \item To compute the first derivative enter the expression \command{f'(x)} in the \field{Input Bar}.
            \item To compute the second derivative enter the expression \command{f''(x)} in the \field{Input Bar}.
            \item To compute the third derivative enter the expression \command{f'''(x)} in the \field{Input Bar}.
            \item To compute the fourth derivative enter the expression \command{f''''(x)} in the \field{Input Bar}.
            \item Thus, the derivative of order $n$ will be $f^n(x)=a^x\log(a)^{n+1}$.
            \end{enumerate}
            \end{indication}
      \item  $g(x)=\dfrac{\sin x +\cos x}{2}$.
            \begin{indication}
            \begin{enumerate}
            \item Enter the function \command{g(x):=(sin(x)+cos(x))/2} in the \field{Input Bar} of the \field{CAS View} of the \field{CAS View}.
            \item To compute the first derivative enter the expression \command{g'(x)} in the \field{Input Bar}.
            \item To compute the second derivative enter the expression \command{g''(x)} in the \field{Input Bar}.
            \item To compute the third derivative enter the expression \command{g'''(x)} in the \field{Input Bar}.
            \item To compute the fourth derivative enter the expression \command{g''''(x)} in the \field{Input Bar}.
            \item From this order on the derivatives are repeated. Thus, the derivative of order $n$ will be
                  \[
                  f^n(x)=
                  \begin{cases}
                  \frac{\sin(x)+\cos(x)}{2}  & \mbox{si $x=4k$}   \\
                  \frac{\cos(x)-\sin(x)}{2}  & \mbox{si $x=4k+1$} \\
                  \frac{-\sin(x)-\cos(x)}{2} & \mbox{si $x=4k+2$} \\
                  \frac{-\cos(x)+\sin(x)}{2} & \mbox{si $x=4k+3$} \\
                  \end{cases}
                  \quad \mbox{with $k\in \mathbb{Z}$}
                  \]
            \end{enumerate}
            \end{indication}
            % \item  $h(x)=\dfrac{1}{\sqrt{1+x}}$.
      \end{enumerate}

\item Compute the tangent line to the graph of the function $f(x)=\log(\sqrt{x+1})$ at $x=1$.
Plot the graph of the function and the tangent line.
      \begin{indication}
      \begin{enumerate}
      \item Enter the function \command{f(x):=log(sqrt(x+1))} in the \field{Input Bar} of the \field{CAS View}.
      \item To compute the equation of the tangent line to $f$ at $x=1$ enter the equation \command{y=f(1)+f'(1)(x-1)} in the \field{Input Bar}.
      \item To plot the tangent line click the circle to the left of the previous expression.
      \item To compute the equation of the normal line to $f$ at $x=1$ enter the equation \command{y=f(1)-1/f'(1)(x-1)} in the \field{Input Bar}.
      \item To plot the normal line click the circle to the left of the previous expression.
      \end{enumerate}
      \end{indication}


\item Consider the function
      \[
      g(x)=\dfrac{2x^{3}-3x}{x^{2}+1}.
      \]

      \begin{enumerate}
      \item Plot the graph of $g$.
            \begin{indication}
            To plot the graph, enter the function \command{g(x):=(2x\^{}3-3x)/(x\^{}2+1)} in the \field{Input Bar} of the \field{CAS View} and activate the \field{Graphics View}.
            \end{indication}

      \item Compute the first derivative of $g$ an plot its graph.
            \begin{indication}
            \begin{enumerate}
            \item To compute the the derivative of $g$ enter the expression \command{g'(x):=Derivative(g)} in the \field{Input Bar}.
            \item To plot the graph of $g'$ click the circle to the left of the previous expression.
            \end{enumerate}
            \end{indication}

      \item Compute the critical points of $g$.
            \begin{indication}
            \begin{enumerate}
            \item To compute the zeros of $g'$ enter the expression \command{Root(g'(x))} in the \field{Input Bar} and click the button \button{Numeric Evaluation}.
            \item $g$ has two critical points at $x=-0.56$ and $x=0.56$ approximately.
            \end{enumerate}
            \end{indication}

      \item Study the growth of $g$ and determine its relative extrema.
            \begin{indication}
            \begin{enumerate}
            \item $g'(x)>0$ for $x\in (-\infty, -0.56)$, so $g$ increases in that interval.
            \item $g'(x)<0$ for $x\in (-0.56, 0.56)$, so $g$ decreases in that interval.
            \item $g'(x)>0$ for $x\in (-0.56, \infty)$, so $g$ increases in that interval.
            \item $g$ has a maximum at $x=-0.56$ since $g'$ vanishes at that point and is positive to the left and negative to the right of this point.
            \item $g$ has a minimum at $x=0.56$ since $g'$ vanishes at that point and is negative to the left and positive to the right of this point.
            \end{enumerate}
            \end{indication}

      \item Compute the second derivative of $g$ an plot its graph.
            \begin{indication}
            \begin{enumerate}
            \item To compute the second derivative of $g$ enter the expression \command{g''(x):=Derivative(g, 2)} in the \field{Input Bar}.
            \item To plot the graph of $g''$ click the circle to the left of the previous expression.
            \end{enumerate}
            \end{indication}

      \item Compute the zeros of the second derivate of $g$.
            \begin{indication}
            \begin{enumerate}
            \item To compute the zeros of $g''$ enter the expression \command{Root(g''(x))} in the \field{Input Bar} and click the button \button{Numeric Evaluation}.
            \item $g''$ has zeros at $x=-\sqrt{3}$, $x=0$ and $x=\sqrt{3}$.
            \end{enumerate}
            \end{indication}

      \item Study the concavity of $g$ and determine its inflection points.
            \begin{indication}
            \begin{enumerate}
            \item $g''(x)>0$ for $x\in (-\infty, -\sqrt{3})$, so $g$ is concave in that interval.
            \item $g''(x)<0$ for $x\in (-\sqrt{3}, 0)$, so $g$ is convex in that interval.
            \item $g''(x)>0$ for $x\in (0, \sqrt{3})$, so $g$ is concave in that interval.
            \item $g''(x)<0$ for $x\in (\sqrt{3}, \infty)$, so $g$ is convex in that interval.
            \item $g$ has inflections points at $x=-\sqrt{3}$, $x=0$ and $x=\sqrt{3}$ since $g''$ vanishes at those points and the sign of the second derivate to the left and to the right of them is different.
            \end{enumerate}
            \end{indication}
      \end{enumerate}

\end{enumerate}


\section{Proposed exercises}
\begin{enumerate}[leftmargin=*]
\item Prove that the following function is not differentiable at $x=0$.
\[
f(x)=
\begin{cases}
e^x-1 & \mbox{if $x\geq 0$};  \\
x^3 & \mbox{if $x<0$}.
\end{cases}
\]

\item For each of the following functions compute the equation of the tangent and normal lines at the points given.
\begin{enumerate}
\item  $y=x^{\sin x},\quad x_{0}=\pi/2$.
\item  $y=(3-x^2)^4\sqrt[3]{5x-4},\quad x_{0}=1$.
\item  $y=\log \sqrt{\dfrac{1+x}{1-x}}+\arctan x, \quad x_{0}=0$.
\end{enumerate}

\item Study the increase, decrease, relative extrema, concavity and inflection points of the function $f(x)=\dfrac{x}{x^2-2}$.

\item A drug has to be given to patients in cylindrical pills.
The content of the drug in each pill is 0.15 ml; determine the dimensions of the cylinder so that the amount of material used to make it (the pill) is minimal.

\item The wheat yield $C$ of a field depends on the level of nitrogen on the ground $n$, and it is given by the following relation:
\[
C(n) = \frac{n}{1+n^2},\quad n\geq 0.
\]
Find the level of nitrogen that will produce the biggest yield.
\end{enumerate}
